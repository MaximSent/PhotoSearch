\newsection
\centertocsection{СПИСОК ИСПОЛЬЗОВАННЫХ ИСТОЧНИКОВ}

%\begin{hyphenrules}{nohyphenation} %отключение переноса слов в содержании
\begin{thebibliography}{9}
    \bibitem{react} Порселло, Б. React. Современные шаблоны для разработки приложений~/ Б. Порселло.~– Санкт-Петербург~: Питер, 2022.~– 320 с.~– ISBN 978-5-4461-1492-4.~– Текст~: непосредственный.
    \bibitem{clearCode}	Мартин, Р. Чистый код. Создание, анализ и рефакторинг~/ Р. Мартин.~– Санкт-Петербург~: Питер, 2020.~– 464 с.~– ISBN 978-5-4461-0960-9.~– Текст~: непосредственный.
    \bibitem{haystack} Haystack : Haystack documentation : сайт. - URL: https://docs.haystack.deepset.ai/docs/intro (дата обращения: 10.04.2024).~– Текст~: электронный.
	\bibitem{vit} Arxiv : Learning Transferable Visual Models From Natural Language Supervision : сайт. - URL: https://arxiv.org/abs/2103.00020 (дата обращения: 12.04.2024).~– Текст~: электронный.
    \bibitem{python} Python : Python 3.12.3 documentation : сайт. - URL: https://docs.python.org/3/index.html (дата обращения: 28.03.2024).~– Текст~: электронный.
    \bibitem{javaScript} Скотт, А. Разработка на JavaScript. Построение кроссплатформенных приложений с помощью GraphQL, React~/ А. Скоттарланд.~– Санкт-Петербург~: Питер, 2021.~– 320 с.~– ISBN 978-5-4461-1462-7.~– Текст~: непосредственный.
    \bibitem{materisalUI} Material UI : Material UI components : сайт. - URL: https://mui.com/material-ui/all-components/ (дата обращения: 14.04.2024).~– Текст~: электронный.
    \bibitem{python3} Хеллман, Д. Стандартная библиотека Python 3. Справочник с примерами~/ Д. Хеллман.~– Москва~: Диалектика, 2020.~– 1376 с.~– ISBN 978-5-6040043-8-8.~– Текст~: непосредственный.
    \bibitem{scikitln} Рашка, С, Мирджалили, В. Python и машинное обучение. Машинное и глубокое обучение с использованием Python, scikit-learn~/ С. Рашка, В. Мирджалили.~– Москва~: Диалектика, 2020.~– 848 с.~– ISBN 978-5-907203-57-0.~– Текст~: непосредственный.
    \bibitem{docker} Docker : Docker  Docs : сайт. - URL: https://docs.docker.com/ (дата обращения: 29.04.2024).~– Текст~: электронный.
    \bibitem{scikitln} Лутц, М. Программирование на Python. Том 1~/ М. Лутц.~– Москва~: Вильямс, 2024.~– 768 с.~– ISBN 978-5-907705-28-9.~– Текст~: непосредственный.
    \bibitem{materisalUI} Flask :Flask documentation : сайт. - URL: https://flask.palletsprojects.com/en/3.0.x/ (дата обращения: 09.04.2024).~– Текст~: электронный.
\end{thebibliography}
%\end{hyphenrules}
