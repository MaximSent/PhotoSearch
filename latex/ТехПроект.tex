\newsection
\section{Технический проект}
\subsection{Общие сведения о программной системе}

Необходимо спроектировать и разработать веб-сервис для контекстного поиска и рекомендации изображений на основе машинного обучения.

Разрабатываемая программная система предназначена для предоставления мощного инструмента для фотографов, дизайнеров, маркетологов, исследователей и компаний, которым необходим эффективный способ доступа и организации визуального контента. Платформа предоставит пользователям возможность загружать, искать и скачивать изображения, а также получать похожие фотографии на основе контекстного поиска.

Основной принцип работы системы заключается в комплексной обработке загруженных изображений для извлечения их метаданных и визуальных характеристик, что позволяет пользователям легко находить нужные изображения по различным критериям. Система будет использовать машинное обучение для улучшения процесса поиска и предоставления персонализированных рекомендаций.

\subsection{Проектирование архитектуры программной системы}
\subsubsection{Выбор архитектурного стиля и паттернов проектирования}

Для разработки веб-сервиса для контекстного поиска и рекомендации изображений на основе машинного обучения была выбрана клиент-серверная архитектура. Этот подход позволяет разделить систему на две основные части: клиентскую (frontend) и серверную (backend), что обеспечивает гибкость в разработке, развертывании и масштабировании компонентов.

Для взаимодействия между клиентом и сервером будет использоваться REST API (Representational State Transfer Application Programming Interface). REST API обеспечивает стандартизированный и легковесный способ обмена данными между различными компонентами системы. Основные принципы REST API включают:

\begin{itemize}
	\item \textbf{Использование HTTP-методов:} REST API использует стандартные HTTP-методы (GET, POST, PUT, DELETE) для выполнения операций над ресурсами.
	\item \textbf{Структурирование URI:} В REST API ресурсы идентифицируются с помощью унифицированных URI (Uniform Resource Identifier). Это позволяет однозначно определять ресурсы и выполнять над ними операции.
	\item \textbf{Безопасность данных:} Все данные, передаваемые между клиентом и сервером, будут защищены с использованием HTTPS. HTTPS обеспечивает шифрование данных, что гарантирует конфиденциальность и целостность передаваемой информации, а также защиту от атак.
	\item \textbf{Формат данных:} REST API поддерживает обмен данными в различных форматах, наиболее распространенным из которых является JSON (JavaScript Object Notation). JSON обеспечивает легкость чтения и обработки данных как на клиентской, так и на серверной стороне.
	\item \textbf{Статусные коды:} В ответах API используются стандартные HTTP-статусные коды для информирования клиентов о результате их запросов.
\end{itemize}

Все данные, передаваемые между клиентом и сервером, будут защищены с использованием HTTPS. Это обеспечивает конфиденциальность, целостность передачи данных и защиту от атак. HTTPS гарантирует, что информация, передаваемая между пользователями и системой, будет защищена от перехвата и подделки.

Для интеграции различных компонентов системы и обеспечения совместимости будет использоваться паттерн проектирования "Адаптер". Этот паттерн позволяет создавать интерфейсы для взаимодействия несовместимых компонентов, обеспечивая гибкость и возможность масштабирования системы. Паттерн "Адаптер" будет применяться для взаимодействия с базами данных и библиотеками.

Архитектура всей системы представлена на рисунке \ref{fig:-architecture}.
\begin{figure}
	\centering
	\includegraphics[width=0.9\linewidth]{"images/Архитектура-системы"}
	\caption{Архитектура системы}
	\label{fig:-architecture}
\end{figure}

\subsubsection{Используемые базы данных}

Для обеспечения хранения и управления данными в веб-сервисе для контекстного поиска и рекомендации изображений на основе машинного обучения будут использоваться следующие базы данных: MinIO и OpenSearch. Эти системы обеспечат высокую производительность, масштабируемость и надежность работы сервиса.

\paragraph{MinIO}

MinIO — это высокопроизводительная объектная система хранения, совместимая с Amazon S3. MinIO предоставляет гибкий и масштабируемый способ хранения больших объемов данных, таких как изображения, видео и другие файлы. Основные особенности использования MinIO в данном проекте включают:

\begin{itemize}
	\item Объектное хранение данных: MinIO позволяет хранить данные в виде объектов, каждый из которых состоит из самого файла, метаданных и уникального идентификатора. Это упрощает управление большими объемами неструктурированных данных.
	\item Высокая производительность: MinIO оптимизирован для высокой производительности и может обрабатывать большие объемы данных с минимальными задержками. Это особенно важно для приложений, работающих с мультимедийным контентом.
	\item Масштабируемость: MinIO поддерживает горизонтальное масштабирование, что позволяет увеличивать емкость и производительность системы по мере роста объема данных. Это достигается за счет добавления новых узлов в кластер MinIO.
	\item Безопасность: MinIO обеспечивает высокий уровень безопасности, включая поддержку шифрования данных как в процессе передачи, так и в состоянии покоя. Это помогает защитить данные пользователей от несанкционированного доступа и утечек.
\end{itemize}

\paragraph{OpenSearch}

OpenSearch — это распределенная поисковая и аналитическая система, основанная на Elasticsearch распространяемая под лицензией Apache-2.0 и являющаяся поисковой системой RESTful с открытым исходным кодом. OpenSearch используется для индексирования и поиска данных, предоставляя мощные возможности для работы с текстовыми и структурированными данными. Основные особенности использования OpenSearch в данном проекте включают:

\begin{itemize}
	\item Индексирование данных: OpenSearch позволяет индексировать большие объемы данных, обеспечивая быстрый и эффективный поиск по различным критериям. Это особенно важно для задач контекстного поиска и рекомендаций изображений.
	\item Полнотекстовый поиск: OpenSearch предоставляет мощные возможности для полнотекстового поиска, включая поддержку сложных запросов, ранжирование результатов и анализ текста. Это позволяет улучшить точность и релевантность результатов поиска.
	\item Аналитика и визуализация: OpenSearch включает встроенные инструменты для аналитики и визуализации данных, такие как OpenSearch Dashboards. Это позволяет создавать интерактивные отчеты и дашборды для анализа данных и принятия обоснованных решений.
	\item Распределенная архитектура: OpenSearch поддерживает горизонтальное масштабирование, что позволяет обрабатывать большие объемы данных и обеспечивать высокую производительность системы. Распределенная архитектура также повышает отказоустойчивость и доступность системы.
	\item Интеграция с другими системами: OpenSearch легко интегрируется с другими системами и инструментами, что позволяет использовать его в различных сценариях и улучшать функциональность веб-сервиса.
\end{itemize}

\subsubsection{Взаимодействие MinIO и OpenSearch}

Использование MinIO и OpenSearch в данном проекте обеспечивает эффективное хранение, управление и поиск данных, что позволяет создавать мощный и удобный инструмент для работы с изображениями. Взаимодействие этих компонентов внутри системы должно происходить следующим образом:

Загрузка и хранение изображений:

\begin{enumerate}
	\item Загрузка изображений: Когда пользователь загружает изображение в систему, оно отправляется на сервер, где происходит его первичная обработка. После этого изображение сохраняется в MinIO, где каждому объекту присваивается уникальный идентификатор и добавляются метаданные, такие как название файла, размер, тип изображения и дата загрузки.
	\item Извлечение и индексирование метаданных: После загрузки изображения в MinIO, изображение обрабатывается нейронной сетью и векторное представление обработанного изображения отправляются в OpenSearch для индексирования. Это позволяет создавать поисковые индексы, которые будут использоваться для быстрого поиска и фильтрации изображений по различным параметрам.
\end{enumerate}

Поиск и рекомендации изображений:

\begin{enumerate}
	\item Запросы на поиск: Когда пользователь выполняет поиск изображений на платформе, запрос отправляется в OpenSearch. OpenSearch использует свои индексы для быстрого нахождения релевантных результатов на основе заданных критериев в векторном пространстве среди всех изображений. Это могут быть текстовые запросы, фильтры по дате, автору, тегам и другим метаданным.
	\item Контекстный поиск и рекомендации: Для улучшения качества поиска и предоставления персонализированных рекомендаций, система будет использовать алгоритмы машинного обучения. Эти алгоритмы анализируют поведение пользователя, его предпочтения и контекст запроса, чтобы предложить наиболее релевантные изображения. Результаты поиска и рекомендации будут динамически обновляться в зависимости от контекста и истории взаимодействий пользователя с платформой.
\end{enumerate}

\subsubsection{Выбор модели нейронной сети}

Для реализации контекстного поиска и рекомендации изображений в веб-сервисе будут использоваться модели нейронных сетей, специализирующиеся на мультиформатном понимании данных. Для данной задачи выбраны две модели: sentence-transformers/clip-ViT-B-32-multilingual-v1 и sentence-transformers/clip-ViT-B-32. Эти модели должны быть совместно интегрированы с помощью фреймворка Haystack для обеспечения высококачественного поиска и рекомендаций.

Модель sentence-transformers/clip-ViT-B-32-multilingual-v1 основана на архитектуре CLIP (Contrastive Language-Image Pre-training), которая была разработана компанией OpenAI. Эта модель позволяет преобразовывать текстовые запросы в векторные представления, что делает её идеальной для задач, связанных с мультиформатным поиском. Модель поддерживает более 50 языков, что позволяет использовать её для текстовых запросов на различных языках в разных странах. Данная нейросеть способна быстро обрабатывать текстовые запросы и преобразовывать их в векторные представления, что ускоряет процесс поиска и улучшает пользовательский опыт при взаимодействии с веб-системой.

Модель sentence-transformers/clip-ViT-B-32 также основана на архитектуре CLIP и предназначена для работы с изображениями. Она используется для преобразования изображений в векторные представления, что позволяет сравнивать их с текстовыми запросами, представленными в векторной форме.

\subsection{Обоснование выбора технологий проектирования и моделей нейронных сетей}
\subsubsection{Выбор используемых технологий и языков программирования}

\paragraph{Python}

Python - это высокоуровневый язык программирования, известный своей простотой и читаемостью. Он широко используется в разработке веб-приложений и машинного обучения благодаря богатой экосистеме библиотек и фреймворков. большое количество библиотек для машинного обучения, такие как TensorFlow или PyTorch, упрощают и ускоряют разработку сложных приложений.

\paragraph{Flask}

Flask - это легковесный веб-фреймворк для Python, который позволяет быстро и просто создавать веб-приложения. Flask не включает много встроенных функций, что делает его гибким и позволяет разработчикам добавлять только необходимые компоненты для реализации API.

\paragraph{React}

React - это JavaScript-библиотека для создания пользовательских интерфейсов, разработанная Facebook. React позволяет создавать быстрые и интерактивные веб-приложения благодаря созданию модульных и повторно используемых компонентов, что упрощает разработку и поддержку кода. Так же React обеспечивает высокую производительность за счет минимизации манипуляций с реальным DOM при обновлении компонентов.

\paragraph{Material-UI}

Material-UI - это библиотека компонентов для React, основанная на принципах Material Design, разработанных Google. Она предоставляет набор готовых, стилизованных компонентов, таких как кнопки, формы, карточки и модальные окна. Все эти компоненты легко настраиваются под потребности конкретного проекта и обеспечивают консистентный и современный внешний вид приложения, что улучшает пользовательский опыт.

\paragraph{Haystack}

Haystack - это фреймворк для построения поисковых систем на основе машинного обучения. Он поддерживает интеграцию с различными моделями и базами данных и обеспечивает высококачественный поиск и рекомендации, включая поддержку мультиформатных данных.

\paragraph{MinIO}

MinIO - это высокопроизводительная объектная система хранения, совместимая с Amazon S3, которая позваляет эффективно хранить и управлять большими объемами неструктурированных данных, обеспечивая быструю обработку данных с минимальными задержками.

\paragraph{OpenSearch}

OpenSearch - это распределенная поисковая и аналитическая система, основанная на Elasticsearch обеспечивающая быстрое и эффективное индексирование больших объемов данных. Так же включает встроенные инструменты для аналитики и визуализации данных, такие как OpenSearch Dashboards.

\subsection{Проектирование пользовательского интерфейса программной системы}
\subsubsection{Макеты пользовательского интерфейса}

На основании требований к пользовательскому интерфейсу, представленных в пункте 2.3 технического задания, был разработан графический интерфейс, используя React с использованием библиотеки Material UI. Разработанный интерфейс ориентирован на обеспечение легкости в использовании и удобного поиска фотографий в системе.

На рисунке \ref{fig:homepage} представлен макет главной страницы.

\begin{figure}
	\centering
	\includegraphics[width=0.95\linewidth]{"images/Главная страница"}
	\caption{Макет главной страницы}
	\label{fig:homepage}
\end{figure}

Макет содержит следующие элементы:
\begin{itemize}
	\item Кнопку добавления новой фотографии.
	\item Поле для ввода запроса.
	\item Кнопку поиска.
	\item Кнопку поиска по фотографии.
\end{itemize}

На рисунке \ref{fig:uploadPage} представлен макет окна добавления файла.
\begin{figure}
	\centering
	\includegraphics[width=0.95\linewidth]{"images/Окно добавления файла"}
	\caption{Макет окна добавления файла}
	\label{fig:uploadPage}
\end{figure}

Макет содержит следующие элементы:
\begin{itemize}
	\item Заголовок страницы.
	\item Поле для перетаскивания файлов (drag-and-drop).
	\item Кнопка для выбора файлов с компьютера.
	\item Прогресс-баром загрузки.
\end{itemize}

На рисунке \ref{fig:galleryPage} представлен макет окна отображения фотографий.
\begin{figure}
	\centering
	\includegraphics[width=0.95\linewidth]{"images/Окно отображения фотографий"}
	\caption{Макет окна отображения фотографий}
	\label{fig:galleryPage}
\end{figure}

Макет содержит следующие элементы:
\begin{itemize}
	\item Поле для поиска фотографий.
	\item Сетка фотографий с предпросмотром.
	\item Блок загрузки новых фотографий.
\end{itemize}

На рисунке \ref{fig:photoPage} представлен макет окна отображения выбранной фотографии.

\begin{figure}
	\centering
	\includegraphics[width=0.95\linewidth]{"images/Окно отображения выбранной фотографии"}
	\caption{Макет окна отображения выбранной фотографии}
	\label{fig:photoPage}
\end{figure}

Макет содержит следующие элементы:
\begin{itemize}
	\item Изображение в полном размере.
	\item Кнопка для скачивания фотографии.
	\item Список похожих на эту фотографий.
\end{itemize}
