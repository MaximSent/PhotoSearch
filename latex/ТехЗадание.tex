\newsection
\section{Техническое задание}
\subsection{Основание для разработки}

Основанием для разработки веб-сервиса для контекстного поиска и рекомендации изображений на основе машинного обучения является задание на выпускную квалификационную работу приказ ректора ЮЗГУ от <<  >>     2024 года № 0000-0 <<Об утверждении тем выпускных квалификационных работ и руководителей выпускных квалификационных работ>>.

\subsection{Назначение разработки}

Функциональное назначение разрабатываемого веб-сервиса заключается в предоставлении пользователям мощного инструмента для поиска и загрузни новых изображений. Платформа предназначена для фотографов, дизайнеров, маркетологов, исследователей и компаниям, которым необходим эффективный способ доступа и организации визуального контента.

Предполагается, что данным веб-сервисом будут пользоваться как профессионалы для управления своими проектами и архивами изображений, так и обычные пользователи, заинтересованные в поиске визуальной информации по введенному запросу.

Задачами разработки данным веб-сервисом являются:
\begin{itemize}
	\item создание обширной и легко доступной базы изображений;
	\item разработка алгоритмов контекстного поиска, позволяющих анализировать и находить изображения по визуальным и текстовым запросам;
	\item разработка функции добавления новых фотографий в базу изображений;
	\item обучение нейронной сети для контекстного поиска.
\end{itemize}

Кроме того, платформа должна выполнять следующие функции:

\begin{enumerate}
	\item Интеграция с социальными сетями и другими платформами: Важно обеспечить возможность интеграции с популярными социальными сетями и другими онлайн-платформами, чтобы пользователи могли легко делиться найденными изображениями или использовать их в своих проектах.
	\item Поддержка различных форматов изображений: Веб-сервис должен поддерживать широкий спектр форматов изображений, обеспечивая их корректное отображение и обработку. Это повысит гибкость использования сервиса для различных нужд пользователей.
	\item Безопасность и конфиденциальность данных: Необходимо гарантировать безопасность и конфиденциальность данных пользователей. Это включает защиту личной информации и обеспечивание надежного хранения загруженных изображений.
\end{enumerate}

Таким образом, данный веб-сервис предоставит пользователям всесторонние возможности для работы с изображениями, делая процесс поиска, управления и использования визуального контента более удобным и результативным.

\subsection{Требования к веб-сервису}
\subsubsection{Требования к данным веб-сервиса}

Входными данными для веб-сервиса являются фотографии со следующими расширениями:

\begin{itemize}
	\item *.jpg;
	\item *.png;
	\item *.webp.
\end{itemize}

Выходными данными для веб-сервиса являются отображаемые фотографии по запросу или рекомендации с возможностью их быстрого скачивания.

На рисунке \ref{fig:-conceptual_classes} представлены концептуальные классы веб-сервиса.
\begin{figure}
	\centering
	\includegraphics[width=0.9\linewidth]{"images/Концептуальные-классы"}
	\caption[Концептуальные классы веб-сервиса]{Концептуальные классы веб-сервиса}
	\label{fig:-conceptual_classes}
\end{figure}

\subsubsection{Функциональные требования к веб-сервису}

На основании анализа предметной области в разрабатываемом веб-сервисе должны быть реализованы следующие функции:
\begin{itemize}
	\item поиск фотографий по запросу;
	\item добавление новой фотографии;
	\item просмотр фотографии;
	\item скачивание выбранного фото;
	\item поиск фото по фотографии;
	\item выдача похожих фотографий.
\end{itemize}

На рисунке \ref{fig:-use_case_diagram} представлены функциональные требования к системе в виде диаграммы прецедентов нотации UML.
\begin{figure}
	\centering
	\includegraphics[width=0.9\linewidth]{"images/Прецеденты"}
	\caption{Диаграмма прецедентов}
	\label{fig:-use_case_diagram}
\end{figure}

\paragraph{Вариант использования «Поиск фотографий по запросу»}
Заинтересованные лица и их требования: Пользователь веб-сервиса, который хочет выполнить поиск фотографии по текстовому запросу.

Предусловие: Пользователь загружает главную страницу веб-сервиса.

Постусловие: Пользователь просматривает предложенные по запросу фотографии.

Основной успешный сценарий:
\begin{enumerate}
	\item Пользователь заходит на главную страницу.
	\item Пользователь вводит запрос в поле поиска.
	\item Пользователь нажимает "Enter" или кнопку поиска.
	\item Система отображает фотографии которые больше всего подходят для данного запроса.
\end{enumerate}

\paragraph{Вариант использования «Добавление новой фотографии»}
Заинтересованные лица и их требования: Пользователь веб-сервиса, который хочет загрузить новую фотографию в базу фотографий.

Предусловие: Пользователь загружает главную страницу сайта.

Постусловие: Пользователь загрузил новую фотографию в веб-сервис.

Основной успешный сценарий:
\begin{enumerate}
	\item Пользователь заходит на главную страницу.
	\item Пользователь выбирает кнопку добавления новой фотографии.
	\item Пользователь загружает новую фотографию.
	\item Система загружает обработанную фотографию и отображает информацию на сайте.
\end{enumerate}

\paragraph{Вариант использования «Просмотр фотографии»}
Заинтересованные лица и их требования: Пользователь веб-сервиса, который хочет просмотреть предложенную фотографию из списка.

Предусловие: Пользователь выполнил поиск фотографии.

Постусловие: Пользователь открывает фотографию для просмотра.

Основной успешный сценарий:
\begin{enumerate}
	\item Пользователь выполнил поиск фотографии.
	\item Пользователь находит понравившуюся фотографию из предложенного списка.
	\item Пользователь нажимает на понравившуюся фотографию.
	\item Выбранная фотография открывает на полный экран.
\end{enumerate}

\paragraph{Вариант использования «Скачивание выбранного фото»}
Заинтересованные лица и их требования: Пользователь веб-сервиса, который хочет скачать найденную фотографию.

Предусловие: Пользователь выполнил поиск фотографии.

Постусловие: Пользователь скачивает фотографию на свой компьютер.

Основной успешный сценарий:
\begin{enumerate}
	\item Пользователь выбирает требуемую фотографию.
	\item Пользователь нажимает кнопку "Сохранить".
	\item Фотография загружается на компьютер пользователя.
\end{enumerate}

\paragraph{Вариант использования «Поиск фото по фотографии»}
Заинтересованные лица и их требования: Пользователь веб-сервиса, который хочет выполнить поиск фотографии по своей фотографии.

Предусловие: Пользователь загружает главную страницу веб-сервиса.

Постусловие: Пользователь просматривает предложенные по запросу фотографии.

Основной успешный сценарий:
\begin{enumerate}
	\item Пользователь заходит на главную страницу.
	\item Пользователь нажимает кнопку камеры в поле поиска.
	\item Пользователь загружает свою фотографию.
	\item Пользователь нажимает "Enter" или кнопку поиска.
	\item Система отображает фотографии которые больше всего подходят для данной фотографии.
\end{enumerate}

\paragraph{Вариант использования «Выдача похожих фотографий»}
Заинтересованные лица и их требования: Пользователь веб-сервиса, хочет просмотреть похожие фотографии к выбранному фото.

Предусловие: Пользователь выполнил поиск по запросу или по фотографии.

Постусловие: Пользователь просматривает похожие фотографии.

Основной успешный сценарий:
\begin{enumerate}
	\item Пользователь выбирает интересующую фотографию.
	\item Пользователь нажимает кнопку "Похожие".
	\item Система отображает похожие фото для выбранной фотографии.
\end{enumerate}

\subsubsection{Требования пользователя к интерфейсу веб-сервиса}

В веб-сервисе должны присутствовать следующие графические интерфейсы взаимодействия с пользователем:

\begin{enumerate}
	\item Страница с полем для поиска фотографий.
	\item Окно для загрузки файлов.
	\item Окно для отображения фотографии.
\end{enumerate}

Все страницы должны иметь адаптивную верстку для взаимодействия с платформой с разных устройств. Каждая страница должна соответствовать основным принципам UI/UX дизайна, что включает в себя:

\begin{enumerate}
	\item Простота и понятность: Интерфейс должен быть интуитивно понятным, чтобы пользователи могли легко ориентироваться и находить нужные функции.
	\item Согласованность: Все элементы интерфейса должны быть согласованы по стилю, цветовой гамме и расположению, что способствует созданию единого визуального образа веб-сервиса.
	\item Эффективность: Время, затрачиваемое пользователем на выполнение задач, должно быть минимизировано. Это можно достичь за счет удобного размещения элементов управления и быстрой реакции системы на действия пользователя.
	\item Доступность: Интерфейс должен быть доступен для всех пользователей, включая людей с ограниченными возможностями.
	\item Адаптивность: Все страницы должны корректно отображаться на различных устройствах и разрешениях экранов. Это включает в себя использование гибких макетов и адаптивной верстки, которые автоматически подстраиваются под размеры экрана.
\end{enumerate}

Кроме того, должны быть предусмотрены уведомления и подсказки для улучшения пользовательского опыта. Система должна предоставлять пользователям своевременные уведомления и подсказки для упрощения работы и информирования о важных событиях.

\subsection{Нефункциональные требования к программной системе}
\subsubsection{Требования к архитектуре}

Веб-сервис должен быть выполнен в клиент-серверной архитектуре с возможностью докеризации для обеспечения гибкости, масштабируемости и легкого запуска.

\subsubsection{Требования к надежности}

При использовании веб-сервиса должна быть обеспечена стабильная работа серверов, а также регулярное создание резервных копий данных. Все возникающие во время работы приложения ошибки должны быть корректно обработаны. Для удобства пользователей ошибки должны выводиться в виде всплывающих окон на сайте, предоставляя пользователю краткую информацию о проблеме.

\subsubsection{Требования к программному обеспечению}

Для реализации клиентской части веб-сервис должен быть использован язык JavaScript, библиотека компонентов Material UI и фреймворк React. 

Для реализации серверной части веб-сервиса должен использоваться язык программирования Python.

\subsubsection{Требования к аппаратному обеспечению}

Для открытия веб-сервиса потребуется устройство с доступом в сеть интернет, а так же любой из доступных браузеров с поддержкой выполнения JavaScript.

Для работы серверной части веб-сервиса необходим сервер на операционной системе Linux с установленным Docker. Так же дисковое пространство не менее 2 Гб, свободная оперативная память в размере не менее 4 Гб, видеокарта с не менее 512 Мб видеопамяти.

\subsubsection{Требования к оформлению документации}

Разработка программной документации и программного изделия должна производиться согласно ГОСТ 19.102-77 и ГОСТ 34.601-90. Единая система программной документации.
