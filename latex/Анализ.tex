\newsection
\section{Анализ предметной области}
\subsection{Описание предметной области}
В современном мире визуальный контент, такой как фотографии и картинки, становится все более значимым и доступным благодаря развитию интернета, социальных сетей и мобильных устройств. 
Однако, с каждым годом количество создаваемых фотографий продолжает стремительно расти. Социальные сети и платформы для обмена медиафайлами, такие как VK, Одноклассники и Instagram, стимулируют пользователей к ежедневному созданию и публикации новых изображений, делая фотографию неотъемлемой частью повседневной коммуникации и самовыражения. 
По оценкам исследований, ежегодно производится несколько триллионов новых фотографий, и этот объем продолжает увеличиваться, что подчеркивает важность развития более эффективных инструментов для управления, хранения и поиска изображений в этом бесконечно растущем массиве визуальной информации. Традиционные методы, такие как простые текстовые запросы и ручной поиск фотографий, не всегда способны полноценно описать то, что человек хочет найти или увидеть за разумное время. В таких случаях важную роль играет контекст, который может включать в себя время, местоположение, социальные связи, а также предпочтения и интересы пользователей.

Контекстный поиск и рекомендация изображений представляют собой подходы, направленные на улучшение процесса поиска и выбора изображений путем учета контекстуальных факторов. Вместо того чтобы рассматривать изображения изолированно, эти системы анализируют контекст, в котором они используются или запрашиваются, и адаптируют результаты под конкретные потребности пользователя.

Для достижения этой цели в области контекстного поиска и рекомендации изображений активно применяются технологии машинного обучения. Методы классификации изображений, извлечения признаков, а также алгоритмы коллаборативной и контентной фильтрации используются для анализа и обработки визуального контента, а также для адаптации результатов под контекст запроса или использования.

\subsubsection{Исторический контекст и эволюция визуального контента}
Исторически визуальный контент всегда играл важную роль в передаче информации и культуры. От наскальных рисунков древних людей до современных цифровых фотографий, изображения служили средством сохранения и передачи знаний, эмоций и художественных идей. С развитием технологий способы создания и распространения изображений значительно изменились.

\textbf{Доцифровой период}

В доцифровую эпоху основными средствами создания визуального контента были живопись, скульптура и графика. Наскальные рисунки, созданные древними людьми, считаются одними из первых форм визуального искусства. Эти изображения позволяли передавать информацию о повседневной жизни, религиозных ритуалах и охоте. В античности и средневековье визуальный контент также играл важную роль в культуре и религии, служа средством записи исторических событий, мифов и легенд.

\textbf{XIX век: изобретение фотографии}

В XIX веке изобретение фотографии революционизировало способ документирования событий и людей. Первая фотография была создана в 1826 году Жозефом Нисефором Ньепсом. Со временем технология совершенствовалась, и фотографии стали важным инструментом в журналистике, науке и искусстве. Фотографии использовались для фиксации исторических событий, научных исследований и создания художественных произведений. Они позволили запечатлеть моменты, которые ранее можно было только описать словами.

\textbf{XX век: развитие кино и телевидения}

В XX веке развитие кино и телевидения добавило новые измерения к визуальному контенту, позволяя передавать движение и звук. Киноиндустрия начала развиваться с изобретением кинематографа братьями Люмьер в 1895 году. Кинофильмы стали популярным средством развлечения и культурного влияния, а также мощным инструментом для передачи историй и идей. Телевидение, появившееся в 1930-х годах, предоставило возможность массового распространения визуального контента, сделав его доступным для широкой аудитории.

\textbf{Конец XX века: цифровая революция}

С появлением цифровых технологий в конце XX века создание и распространение визуального контента стало еще более доступным. Цифровые камеры, а затем и камеры в мобильных устройствах, сделали фотографию повседневной практикой для миллионов людей. Этот период ознаменовался быстрым развитием технологий, таких как компьютерная графика и цифровое редактирование изображений, что значительно расширило возможности создания и обработки визуального контента.

\textbf{XXI век: эра социальных сетей и онлайн-платформ}

В XXI веке социальные сети и онлайн-платформы предоставили новые возможности для обмена и публикации изображений, делая их глобально доступными в считанные секунды. Платформы, такие как Instagram, Facebook, и Pinterest, позволяют пользователям делиться своими фотографиями и видео с широкой аудиторией, получать обратную связь и взаимодействовать с другими пользователями. Это привело к созданию огромных массивов визуального контента, который требует эффективных инструментов для управления, поиска и анализа.

\textbf{Будущее визуального контента}

С развитием технологий, таких как искусственный интеллект, машинное обучение и дополненная реальность, будущее визуального контента обещает быть еще более захватывающим. Эти технологии открывают новые возможности для создания, анализа и взаимодействия с визуальным контентом. Например, искусственный интеллект уже используется для автоматической классификации и поиска изображений, а дополненная реальность позволяет создавать новые формы визуального опыта.

\subsection{Технологии машинного обучения в контекстном поиске и рекомендации изображений}
В последние годы, поиск и рекомендация изображений, значительно продвинулись благодаря использованию технологий машинного обучения. Эти технологии позволяют системам не только распознавать и классифицировать визуальный контент на заранее указанные группы, но и адаптироваться к предпочтениям и контексту пользователя. Основная задача таких систем заключается в обеспечении более глубокого понимания содержания изображений, что позволяет осуществлять более точные и персонализированные рекомендации. Эффективность этих систем зависит от выбора подходящих алгоритмов и архитектур нейронных сетей, которые могут адаптироваться и реагировать на различные контекстные сигналы. На данный момент ключевыми являются три следующие архитектуры:

\begin{itemize}
	\item Convolutional Neural Networks (CNN) - является одной из основных архитектур для анализа изображений. Нейронные сети на данной архитектуре способны выделять ключевые признаки изображений на разных уровнях абстракции, что делает их эффективными для задач классификации, детекции и сегментации объектов. CNN успешно используются для извлечения признаков изображений и создания их векторных представлений.
	
	\item Vision Transformer (ViT) - другая важная архитектура нейронных сетей,, которая была представлена в 2017 году и адаптирована для обработки визуальных данных в 2020 году.  Особенностью ViT является способ обработки данных, который заключается в разбивании изображения на патчи и анализа их с помощью трансформерных блоков, создавая векторные представления изображений.
	
	\item Contrastive Language-Image Pre-training (CLIP) - одна из современных архитектур для анализа изображений, разработанная компанией OpenAI в 2021 году. Она подразумевает обучение визуальных данных под контролем нейронной сети естественного языка. Данная архитектура устраняет разрыв между текстовыми и визуальными данными путем совместного обучения модели на крупномасштабном наборе данных, содержащем изображения и соответствующие им текстовые описания.
	
\end{itemize}

\subsection{Существующие решения}
В области поиска и рекомендации изображений существуют как заграничные, так и российские платформы, предоставляющие возможность поиска изображений, расположенных как на самой платформе, так и в интернете в целом, среди которых выделяются такие крупные компании, как Pinterest, Google и Yandex. Эти платформы демонстрируют различные подходы к индексации, поиску и визуализации визуального контента.

\textbf{Pinterest} - это социальная сеть, основанная на принципах визуального поиска и коллекционирования изображений. Платформа позволяет пользователям создавать и управлять тематическими коллекциями изображений, таких как интерьеры, мода, питание и многое другое. Одной из ключевых особенностей Pinterest является использование современных алгоритмов машинного обучения для рекомендации изображений, основанных на предпочтениях пользователя и визуальном сходстве. Это достигается благодаря разработке собственных технологий по анализу изображений и их контексту, что позволяет предлагать пользователю наиболее релевантные и интересные изображения, расположенные на платформе.

\textbf{Google Картинки} представляет собой мощный поисковик изображений, который позволяет пользователям находить визуальный контент по всему интернету. Система использует сложные алгоритмы индексации и ранжирования, которые включают анализ текстовой информации, связанной с изображением, и контекста страницы, на которой это изображение размещено.

\textbf{Yandex Картинки} также как и Google Картинки, позволяет пользователям находить визуальный контент по всему интернету и фильтровать его на основе заданных параметров. 

Однако, несмотря на наличие российских платформ, таких как Yandex, на текущий момент отсутствуют решения, которые бы позволяли рекомендовать изображения исключительно из собственной базы данных, не выходя за рамки платформы, подобно тому, как это делает Pinterest. В дополнение к потребностям общего пользователя, такая система могла бы найти применение в корпоративной сфере, облегчая хранение и поиск рабочих файлов в рамках группы или организации. Это становится особенно важным в контексте увеличения объемов цифровых данных и необходимости их структурирования для удобства доступа и использования. Внедрение системы, ориентированной на работу с внутренней базой данных изображений, может значительно повысить эффективность работы команд, ускоряя процессы поиска необходимых материалов и обеспечивая более тесное взаимодействие в рамках коллективных проектов.
